\chapter{Entwurfsentscheidungen}
\section{Persistenz}
Die Art und Weise, wie Daten beständig und dauerhaft in dem System gespeichert werden können ist mit das Herzstück eines Condition Monitoring Systems. Das Hauptaugenmerk dabei ist es diese Daten in einer besonders hohen Geschwindigkeit zu speichern, da ein besonders hohes Datenvolumen möglich und vom Anwender gewünscht ist. Speichermodelle mit langen Speicherzeiten sind den Anforderungen demnach ungenügend und eignen sich nicht für die Speicherung von Daten. Wide-Column-Store Datenbanken würden sich als mögliches Speichermodell anbieten. Sie sind eine Erweiterung von Key-Value-Store Datenbanken und speichern ihre Daten in einer Struktur, ähnlich dem Schema einer Tabelle ab. Der Schlüssel ist ein Bytearray und der Wert besteht aus Paaren von Spaltenname und -werten. Im Gegensatz zum relationalen Modell gibt es kein fixes Schema. Ein solches System ist zum Beispiel Hypertable, das mit einem geringen Speicherplatzbedarf sich optimal eignet. Nach eigenen Angaben soll das System sich sogar um die Konsistenz der Daten kümmern, während sich viele NoSQL-Datenbanken mit einer vermeintlichen Konsistenz der Daten begnügen.
\subsection{NoSQL-Datenbanksystem}
Im Vergleich standen drei Datenbanksysteme desselben Typs als Vergleich gegenüber. Das Datenbanksystem von Orcale Berkeley DB gibt es schon seit 1994 und wurde je nach Edition in C, C++ oder Java implementiert. Es handelt sich dabei um einen Key-Value Speicher, der XML-Unterstützung bietet, SQL und sekundäre Indexierung kann. Die Datenbank wird nach dem ACID-Prinzip gefüllt und ist trotzdem noch sehr schnell bzw. und mit der richtigen Konfiguration auch komplett parallel nutzbar.
Das Redis ist eine beliebte Datenbanksoftware, die seit 2009 auf dem Markt ist. Sie ist ein in C geschriebener Key-Value-Speicher, der eine große, aber keine vollständige Konsistenz der Daten verspricht. Dafür ist der Zugriff komplett parallel und kann durch einige Einstellungen auch stark konsistent werden. Allerdings ist Konsistenz eine starke Anforderung und verpflichtend für die Datenbank. Das wie eben schon erwähnte Projekt Hypertable wurde im März 2016 eingestellt und basiert auf BigTable und sowie Daten. Es wurde in C++ geschrieben und ist ein schemafreies System. Die Vorteile liegen in der sofortigen Konsistenz der Daten und der Möglichkeit gleichzeitig lesen und schreiben zu können. Der große Nachteil ist, dass es keine Updates mehr gibt, da das Projekt nicht fortgesetzt wird. Trotzdem ist diese Software in den Schreibvorgängen am schnellsten, wie folgendes Benchmark zeigt \url{https://wiki.volution.ro/Dehems/Benchmarks/Results}.
\subsection{SQL-Datenbanksystem}
Für die anderen Datensätze wie Benutzer-, Rechte- und Komponentenverwaltung haben wir uns für SQLite entschieden. Die Datenbank benötigt für den Betrieb äußerst wenig Ressourcen und Leistung und benötigt so beispielsweise gerade mal ca. 350 kByte an Speicher. Hinzu kommt, das durch die einfache Integration und Verwaltung das System schnell integriert und verteilt werden kann. Für ein Benchmark ist folgendes Seminar in Datenbanksysteme der FHO HSR herangezogen worden \cite{MorierWeber2011}.
\subsection{Serialisierung Capn Proto}
Da wir mit einem System arbeiten, das eingehende wie ausgehende Daten, in dem System fachterminologisch als Datenpunkt bezeichnet, verarbeitet und mit programminternen sowie /-externen Einheiten kommuniziert, muss bekanntlich auch eine Serialisierung dieser Daten erfolgen.  Bei einer Reihe zur Verfügung stehender Serialisierungstechnologien ist es für das Produkt nötig die Vor- und Nachteile der jeweils einzelnen abzuwägen. Da es für das System von großer Wichtigkeit ist mit einem möglichst großen Datendurchsatz performant umgehen zu können, können textbasierte Serialisierungstechniken wie JSON, XML, YAML usw. kategorisch abgelehnt werden. Der Grund dafür ist, dass im Allgemeinen ein zu großer Speicherbedarf benötigt wird, was zur Folge hat, dass die Serialisierungszeit und damit auch die Laufzeit stark ansteigt.\\ 
Eine Möglichkeit die Laufzeit zu verringern wäre eine adhoc single String Technik zu verwenden. Diese muss jedoch redundant für jede Sprache geschrieben werden, was uns in der Portierbarkeit einschränken würde.\\
Die nach unserer Meinung nach beste Lösung für dieses Problem wäre es demzufolge eine schemagesteuerte binäre Serialisierungstechnik zu verwenden. Zum Einsatz soll Capn Proto verwendet werden. Durch dessen flexible Schemadefinition lässt sich im Vergleich zu den anderen Portierbarkeit erhöhen, da dieses Framework den Serialisierungscode anhand der Schemadefinition selbst erzeugt. Die binäre Darstellung der Daten erlaubt zudem eine deutlich effizientere Übertragung der Daten. Capn Proto ist im Vergleich zu Protobuf fast doppelt so schnell, siehe \url{https://github.com/ChrisMacNaughton/proto_benchmarks/}. Die Bibliotheken sind einfach zu nutzen und in C++ geschrieben.
\subsection{Parallelisierung}
Damit alle Systeme und Komponenten zeitgleich ihre Aufgaben erfüllen können, wird die Software in Micro Services aufgebaut, die unabhängig voneinander parallel arbeiten können.\\
Weitere Vorteile liegen in der unabhängigen Entwicklung, Implementierung und Wartung der einzelnen Bausteine. Allerdings wird dadurch auch ein Load Balancer nötig, der die Ressourcen der Hardware nach den Anfragen auf die einzelnen Prozesse verteilt. 
\subsection{Fehlererkennung}
Pings oder ICMP Request/Replies werden zwar oft genutzt um zu sehen, ob ein Server noch online ist, aber da ICMP betriebssystemspezifisch implementiert ist und in manchen Sicherheitsrichtlinien deaktiviert sein muss, kann man das hier nicht für eine dauerhafte Taktik verwenden. Außerdem erhält man keine Einsicht darüber, ob ein Programm auf der Maschine ausgeführt ist, was in unserem Fall aber wichtig wäre. 
\\
Mit Hilfe von Monitoring kann man die Antwortzeiten und die Uptimes von allen Anwendungen eines Servers abfragen und darstellen. Diese Methode ist zielführender, braucht allerdings eine zusätzliche Schnittstelle, die nicht immer bereitgestellt werden kann. Auch verbraucht sie einige Ressourcen und die Abfrage kann bzw. sollte nicht in zu kurzen Zeitintervallen stattfinden. 
\\
Heartbeat ist eine periodischer Nachrichtenaustausch zwischen einem Prozess und einem Kontrollsystem. Da unsere Anwendung in Echtzeit große Datenmengen bewältigen muss, ist es wichtig, dass der Heartbeat schnell und einfach gesendet und empfangen wird um keine anderen Daten/Datenverarbeitungen zu behindern. 
Mit Hilfe von Condition Monitoring kann man zum Beispiel bei einer Maschine aus Sensordaten Schwingungen und Temperaturen erfassen, die evtl. für Ausfälle sorgen. Um das bei unserem Server anzuwenden, muss man die CPU-Temperatur überwachen und ggf. den Serverschrank weiter runterkühlen um Ausfälle zu vermeiden. 
Voting ist nicht nur prozessorlastig, sondern wird am besten auch auf mehreren Geräten gleichzeitig ausgeführt was langsam und kostenintensiv ist, daher verwerfen wir diese Möglichkeit.
\\
Zeitstempel kann man einfach in die Datenbank einpflegen um evtl. fehlerlastige Zeiten herauszufinden und zu löschen oder zu korrigieren. Eine größere Validierung wird wieder zu CPU-lastig und kann daher nicht direkt beim Speichern der Daten durchgeführt werden.\\
Aus Kostengründen müssen wir auch alle Wiederherstellungsmaßnahmen, die Redundanz beinhalten leider ausschließen, da das Budget zu klein gesetzt wurde. Daraus folgt, treten Softwarefehler ein, muss man mit Hilfe von Hotfixes und bei Hardwarefehlern mit einem Austauschgerät arbeiten. Außerdem kann man bei Exceptions, die zur Laufzeit auftreten den jeweiligen Stacktrace mit der dazugehörigen Eingabe an eine Kontrollstelle abspeichern. Sobald eine kritische Anzahl an Exceptions pro Minute/Stunde auftritt muss der Fehler in der Software gepatcht werden. Im Softwaredesignprozess muss dabei darauf geachtet werden, dass Exceptions immer gehandelt werden und das auch in sehr ungewöhnlichen oder gar unmöglichen Fällen um in jedem Szenario handlungsfähig zu bleiben.
\\
Abschließend haben wir uns für eine Kombination aus Condition Monitoring, Exception Handling und Zeitstempel entschieden, weil diese kaum Leistung benötigen und Kosten verursachen. Hotfixes für Ausfälle sind gerade im Pilotbetrieb erstmal günstiger als redundante Hardware und komplexe Monitoringtools. Diese können bei Bedarf in einer späteren Variante auf besserer Hardware eingebaut werden.
