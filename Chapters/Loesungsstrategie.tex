\chapter{Lösungsstrategie}
\label{ch:Lösungsstrategie}
In diesem Kapitel werden nun die grundlegenden Entscheidungen und Lösungsansätze unseres Entwurfs vorgestellt. In den folgenden Technologieentscheidungen werden die Entscheidungen der Architektur festgelegt.  
\begin{itemize}
	\item Entkopplung der Datenbanken zur Fachdomäne mittels Schnittstellenklassen und der Nutzung von zwei separaten Datenbanken (Qualitätsziel: Performance)
	\item Verringerung der Flexibilität, durch Verzicht auf Abstraktionsschichten (Qualitätsziel: Performance)
	\item Zyklische redundante Speicherung ausgewählter Daten und Bereitstellung von Hardware-Backupsystemen (Qualitätsziel: Verfügbarkeit)
	\item Einsatz hardwarenaher Programmiersprachen C und C++ (Qualitätsziel: Performance)
	\item Keine Verteilung des Systems(Qualitätsziel: Performance)
	\item Fehlererkennung durch kontinuierliches Versenden von Heartbeat (Qualitätsziel: Verfügbarkeit)
	\item Systemzustand wird periodisch gespeichert um im Fehlerfall zurücksetzen zu können (Qualitätsziel: Verfügbarkeit) 
	\item Benutzerschnittstelle soll Datenobjekte durch Masken erzeugen und manipulieren können (Qualitätsziel: Nutzbarkeit) 
	\item Die Benutzerschnittstelle läuft auf der Zielplattform in einem Web-Client und wird über den nginx-Webserver verwaltet (Qualitätsziel: Nutzbarkeit)
	\item Nutzung von Capt'n Proto beim Serialisieren und Deserialisieren von Daten (Qualitätsziel: Performance)
\end{itemize}