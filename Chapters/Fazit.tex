\chapter{Fazit}
\label{ch:Fazit}
Ziel der vorliegenden Ausarbeitung war es, ein umfangreiches Condition Monitoring System zu planen. Die allgemeinen Vorgaben zu dem Projekt bekamen wir von der Firma HYDAC Systems und Services GmbH mitgeteilt, sodass wir diese nicht komplett erarbeiten mussten. Dabei mussten wir uns mit vielen unterschiedlichen Technologien und Ansätzen vertraut machen um die bestmöglichen Designentscheidungen treffen zu können.\\
Dabei standen vor allem Programme und Schnittstellen im Vordergrund, die sowohl äußerst leistungsfähig, als auch kostenlos verfügbar sind, denn sowohl die Anforderung von 500 Datenpunkten pro Sekunde, als auch das sehr strikte Budget, ließen keinen Spielraum offen. Trotzdem war es uns möglich auf jeder Modellebene für jeden Microservice die entsprechenden Bibliotheken zu finden. Dafür mussten wir in jedem Fall die Vor- und Nachteile miteinander vergleichen und dann abwägen. Dass zwischendurch Aufgaben und Anforderungen missverstanden wurden, führte schon früh zu Verzögerungen in dem Ausarbeitungsprozess, die wir bis zur Abgabe noch bemerkten. Auch der Ausbruch des Coronavirus hat uns zeitlich stark eingeschränkt, da wir in den IT-Abteilungen unserer Unternehmen Überstunden und Mitarbeiterbesuche machen mussten. Ohne diese Verzögerungen hätte unser Ergebnis und unserer Projekt noch besser sein können.\\
Ob das Ergebnis unserer Arbeit wirklich so funktioniert, wie wir es hoffen, muss es bei der Firma HYDAC getestet werden. Im Vergleich zur jetzigen Situation und Architektur sehen wir viele Vorteile in unserer Variante und sind uns sicher, dass wir den Leistungsanforderungen entsprechen. Wann wir den tatsächlichen Vergleich und die dazugehörigen Ergebnisse vorliegen haben, kann man zur Zeit noch nicht sagen.\\
Bei einer erneuten Durchführung des Projekts würden wir verschiedene Dinge ändern, wie zum Beispiel uns öfter zu treffen und die Ergebnisse aus den Treffen auch direkt in die Präsentationen und Abgaben einzupflegen. Außerdem müssen die Gruppenmitglieder bei Verständnisfragen früher den Rest des Teams mit einbeziehen um Folgefehler, die erst spät auffallen und dann nur schwer korrigierbar sind, zu vermeiden. Doch auch trotz aller Probleme und Hindernisse, war es interessant und spannend eine Softwarearchitektur auszuarbeiten.


