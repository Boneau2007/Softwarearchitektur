\chapter{Einleitung}
\label{ch:Einleitung}


\section{Problemstellung}
Die Firma HYDAC Systems und Services GmbH entwickelt derzeit ein System zum erfassen, verarbeiten und auswerten von Messwerten und Prozessdaten in Echtzeit. Allgemein ist solch ein System auch unter dem Begriff Condition Monitoring bekannt und ermöglicht es Prozesse zu überwachen, zu planen und gegebenenfalls zu optimieren. Es wird demnach zur Prozessoptimierung verwendet und erhöht die Verfügbarkeit von Anlagen. Im Rahmen der Machbarkeit und Untersuchung unterschiedlicher Herangehensweisen, ist es für die Firma interessant wie Studenten der Hochschule htwsaar ein Condition Monitoring System anhand gegebener Anforderungen konzipieren würden.

\section{Zielsetzung}
Das Ziel dieser Arbeit soll es sein, einen Architekturentwurf für ein Condtion Monitoring auszuarbeiten, welcher unter Anwendung moderner Architekturansätze zu gestalten ist. Wichtig ist es auch, das die Vorgaben der Firma mit in das Konzept fließen.

\section{Stakeholder}

\section{Gliederung}
Das erste Kapitel gibt einen Überblick über die Themen, welche folgend in der Dokumentation ausformuliert sind. Im Kapitel \ref{ch:Anforderungsanalyse}, werden sowohl die funktionalen, als auch die nicht-funktionalen definiert und erörtert. Es schafft Klarheit darüber wer, warum und wozu solch ein System benötigt wird und definiert die zentralen Merkmale. Das \ref{ch:Konzept} befasst sich danach mit den aus Kapitel \ref{ch:Anforderungsanalyse} gestellten Anforderungen und beschreibt ein mögliches Konzept nach Kriterien. Kapitel Im letzten Kapitel \ref{ch:Fazit} werden wir das Vorgehen abschließend beurteilen und einen Ausblick wagen.